\documentclass{article}
\usepackage[T1]{fontenc}

\usepackage[utf8]{inputenc}
\usepackage[english]{babel}
\usepackage{upgreek}
\usepackage{amsmath}
\usepackage{color}
\usepackage{listings}
\usepackage{graphicx}

\begin{document}
  \title{Hw08 - Report - Distributed Systems}
  \date{}
  \author{Lukas Dötlinger, 01518316}
	
  \maketitle
  
  \section*{Indroduction}
  
  \section*{Part 1)}
    \subsection*{a)}
  
    Availability is calculated using the following formula:\\
    \begin{equation}
      A = \frac{MTBF}{MTBF + MTTR}
    \end{equation}
    \textit{MTBF} ... \textbf{M}ean \textbf{T}ime \textbf{B}etween \textbf{F}ailure\\
    \textit{MTTR} ... \textbf{M}ean \textbf{T}ime \textbf{T}o \textbf{R}epair\\
    \newline
    \textbf{System-A:}\\
    Fails on average three times an hour. This results in an MTBF of $60/3 = 20$ min. Since the failure lasts about 30 seconds, we have an MTTR of $0.5$ min.\\
    \[ A_{SA} = \frac{20}{20+0.5} = 0.\overline{97560} \approx 97.56 \% \]\\
    \newline
    \textbf{System-B:}\\
    Fails on average thirty times an hour. This results in an MTBF of $60/30 = 2$ min. Since the failure lasts about 3 seconds, we have an MTTR of $0.05$ min.\\
    \[ A_{SB} = \frac{2}{2+0.05} = 0.\overline{97560} \approx 97.56 \% \]\\
    
    \newpage
    \subsection*{b)}
    
    Availability for multiple redundant systems is calculated using the following formula:\\
    \begin{equation}
      A_{n*SX} = 1 - (1 - A_{SX})^{n}
    \end{equation}
    \textit{n} ... number of redundant systems\\
    $A_{SX}$ ... availability of one System-X\\
    \newline
    \textbf{System-A:}\\
    We already know that availability for a system A is $97.56 \%$.\\
    \[ A_{2*SA} = 1 - (1 - 0.\overline{97560})^{2} = 0.9994 \approx 99.94 \% \]
    So we need at least 2 redundant systems A, to reach an availability of over 99.9\%.\\
    \newline
    \textbf{System-A with different MTBF:}
    Fails on average three times in 20 minutes. This results in MTBF of $20/3 = 6.\dot{6}$ min. Since the failure lasts about 30 seconds, we have an MTTR of $0.5$ min.\\
    \[ A_{SA} = \frac{6.\dot{6}}{6.\dot{6}+0.5} \approx 93.02 \% \]
    \[ A_{2*SA} = 1 - (1 - 0.9302)^{2} = 0.99513 \approx 99.51 \% \]
    \[ A_{3*SA} = 1 - (1 - 0.9302)^{3} = 0.99966 \approx 99.96 \% \]
    So we need at least 3 redundant systems A, to reach an availability of over 99.9\%.\\
    
  \section*{Part 2)}
    \subsection*{a)}
    
      Version 1 is implemented in the \texttt{Version1.java} file, which can be found in the \texttt{practical/} directory. The input parameters are specified as constants at the top of the class.\\
      The implementation uses a Map with length equals \textit{T}. At each of the N iterations over the map, a probability-function, defined in \texttt{Utils.java} determines if an access was successful. At the end, the successful accesses are divided by the total amount, to get the probability for N redundant systems.\\
      \newline
      Sample output for N = 2 and failure-rate = 0.9756 :
      \begin{lstlisting}[language=sh]
        Availability: 0.9996
      \end{lstlisting}
      This is about what we expect from the theoretical calculation in subsection b) of Part 1). \textbf{Keep in mind that this is a sample output! Since we have some part of randomness in the program, the result may change by a small margin for each run.}\\
      
    \subsection*{b)}
    
      Version 2 is implemented in the \texttt{Version2.java} file, which can be found in the \texttt{practical/} directory. The input parameters are specified as constants at the top of the class.\\
      Version2 is based on Version1, the difference though is that we have a given target-probability, for which we want to find the smallest possible N, instead of a given N. The same calculation for the availability as in the previous version is used inside a while loop. After each run, the N for testing is increased by one. Once the calculated probability is grater or equal the target, N is found.\\
      \newline
      Sample output for target-probability = 0.999 and failure-rate = 0.9302 :
      \begin{lstlisting}[language=sh]
        Availability 0.999 is reached with N >= 3
      \end{lstlisting}
      This is about what we expect from the theoretical calculation in subsection b) of Part 1).
  
\end{document}