\documentclass{article}
\usepackage[T1]{fontenc}

\usepackage[utf8]{inputenc}
\usepackage[english]{babel}
\usepackage{upgreek}
\usepackage{amsmath}
\usepackage{color}
\usepackage{listings}
\usepackage{graphicx}

\begin{document}
  \title{Hw07 - Report - Distributed Systems}
  \date{}
  \author{Lukas Dötlinger, 01518316}
	
  \maketitle
  
  \section*{Introduction}
  
  	To run the replica-management, simply execute the main method of the\\ \texttt{ReplicaManagement.java} file.\\
  	
  \section*{Implementation}
  
    Variables like \textit{w1}, \textit{w2} or \textit{replicaCost} can be changed in the \texttt{ReplicaManagement} class. They are given as \texttt{final} variables at the top.\\
    In the \texttt{main} method, a list of data-centers is created. Afterwards the connections between them and their latencies are initialized using an algorithm on my matricular number.\\
    \textbf{On each run, the requests, which a data-center sends to others, are randomly generated, while the total amount for one data-center to send being within a range of zero and fife.} Based on these requests, the total latencies are calculated. If for example A sends 4 requests to B, which would have a higher total latency than making a replica and accessing that, a new replica is added to the optimized version. I used following formula for that:
    \begin{equation}
      \begin{split}
        if ( & (latency+1) * reqSize > latency + reqSize + repCost) \\
        & create-replica
      \end{split}
    \end{equation}
    After all optimal replicas were found, the total latencies are re-calculated. If replicas were made, the new total cost should be lower than the initial one.
  
\end{document}