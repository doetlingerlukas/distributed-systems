\documentclass{article}
\usepackage[T1]{fontenc}

\usepackage[utf8]{inputenc}
\usepackage[english]{babel}
\usepackage{upgreek}
\usepackage{amsmath}
\usepackage{color}
\usepackage{listings}
\usepackage{graphicx}

\begin{document}
  \title{Hw03 - Report - Distributed Systems}
  \date{}
  \author{Lukas Dötlinger, 01518316}
	
  \maketitle
  
  \section*{Introduction}
    
    Both parts of the homework sheet are completed.\\
    There is a subsection at the end of every part, which contains information on how to run the code. There is also a sample output of the program given.\\
	
  \section*{Part 1)}
  
    \subsection*{a,b)}
    
      The interface \texttt{RemoteService.java} defines the two methods, \textit{sort} and \textit{compute}, and extends \texttt{Remote}. The methods are implemented by the \texttt{RemoteServiceImpl.java} class. The \texttt{Server.java} uses an instance of \texttt{RemoteServiceImpl} to generate a \texttt{RemoteService} \texttt{stub}, which it registers to the \texttt{Registry} under the name \textit{"remoteService"}. The \texttt{Client.java} class starts six clients, three for each method. Every client looks-up the \textit{remoteService} in the registry and uses one of it's method as an RMI call.
      
	\subsection*{c)}
	
	  Yes, the server is capable of handling multiple requests from multiple clients. This is shown in the sample output at the next subsection.\\
	  In comparison to homework 2, it seems to be just as fast. The output for several clients seems to be printed to the console almost at the same time. 
    
    \subsection*{run part1)}
    
      To execute the first part, do the following:
      \begin{itemize}
    	\item[1.] Navigate to the \texttt{part1/} directory an run \texttt{javac *.java} to compile the classes
    	\item[2.] Navigate back to the source directory
    	\item[3.] Start the registry with \texttt{start rmiregistry}
    	\item[4.] Start the server with \texttt{java hw03.part1.Server}
    	\item[5.] Open a second window and start the client with \texttt{java hw03.part1.Client}
      \end{itemize}
      Sample output:\\
      \textbf{Server}:\\
      \begin{lstlisting}[language=sh]
        $ java hw03.part1.Server
        Successfully registered remote service!
      \end{lstlisting}
      \textbf{Client}:\\
      \begin{lstlisting}[language=sh]
        $ java hw03.part1.Client
        Server starts computing for client 3
        Server starts computing for client 1
        Server starts computing for client 2
        88akkknnqt for client 5
        2467flpqrz for client 4
        47889llluz for client 6
        Server ended computing for client 3
        Server ended computing for client 1
        Server ended computing for client 2
      \end{lstlisting}  
      
  \section*{Part 2)}
  
    Part 2 uses a proxy to call the right remote method for a client. Two servers, \texttt{SortServer.java} and \texttt{ComputeServer.java}, each of which implements a different service. Those servers implement the \texttt{Runnable} interface, so they can be started from another class. The proxy starts two instances for each of the two servers. On construction, a server takes an id and on start it registers it's service to the registry. The proxy also registers it's method \texttt{accessService()} to the registry. This method is invoked by a client, which will get the proxy to invoke one of the servers methods. The proxy finds the services in the registry by getting all names from the registry, \texttt{registry.list()}.\\
    \texttt{ProxyClient.java} starts six clients, three for sorting and 3 for computing. The proxy randomly forwards the requests to a correct endpoint. 
    
    \subsection*{run part2)}
    
      To execute the second part, do the following:
      \begin{itemize}
        \item[1.] Navigate to the \texttt{part2/} directory an run \texttt{javac *.java} to compile the classes
    	\item[2.] Navigate back to the source directory
    	\item[3.] Start the registry with \texttt{start rmiregistry}
    	\item[4.] Start the server with \texttt{java hw03.part2.ProxyServer}
    	\item[5.] Open a second window and start the client with \texttt{java hw03.part2.ProxyClient}
      \end{itemize}
      Sample output:\\
      \textbf{ProxyServer}:\\
      \begin{lstlisting}[language=sh]
        $ java hw03.part2.ProxyServer
        Successfully registered sort service 2
        Successfully registered compute service 2
        Successfully registered proxy server!
        Successfully registered compute service 1
        Successfully registered sort service 1
        computeService2 starts computing for client 2
        computeService1 starts computing for client 3
        computeService1 starts computing for client 1
      \end{lstlisting}
      \textbf{ProxyClient}:\\
      \begin{lstlisting}[language=sh]
        $ java hw03.part2.ProxyClient
        1iksvwxxyy sorted by sortService2 for client 5
        hjmprrstyz sorted by sortService1 for client 4
        78bbhhkoyz sorted by sortService1 for client 6
        computeService2 ended computing for client 2
        computeService1 ended computing for client 3
        computeService1 ended computing for client 1
      \end{lstlisting}
	
\end{document}