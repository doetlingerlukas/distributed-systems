\documentclass{article}
\usepackage[T1]{fontenc}

\usepackage[utf8]{inputenc}
\usepackage[english]{babel}
\usepackage{upgreek}
\usepackage{amsmath}
\usepackage{color}
\usepackage{listings}
\usepackage{graphicx}

\begin{document}
  \title{Hw02 - Report - Distributed Systems}
  \date{}
  \author{Lukas Dötlinger, 01518316}
	
  \maketitle
  
  \section*{Introduction}
    
    All three parts of the homework sheet are completed.\\
    There is a subsection at the end of every part, which contains information on how to run the code. There is also a sample output of the program given.\\
	
  \section*{Part 1)}
	
    The first part is implemented using three Java classes, which are inside the \texttt{/part1} directory:
    \begin{itemize}
      \item \texttt{SimpleClient.java} represents a single client thread.
      \item \texttt{SimpleClientExecutor.java}, a class to start and execute client threads.
      \item \texttt{MultithreadedSortingServer.java} represents the server.
    \end{itemize}
    \texttt{SimpleClient.java} implements the \texttt{Runnable} interface and can therefore be started as thread. It has three private variables: \texttt{int port} the servers port, \texttt{String server} the address of the server and \texttt{String toSort} the string the client wants to be sorted by the server. When construction an instance of this class, all variables have to be given. Since it implements the \texttt{Runnable} interface, it has a \texttt{void run()} method, which calls the \texttt{void request()} method. The request method is the main action of the thread. It gets the servers \texttt{Socket} by it's address and port. After that it opens the input/output-streams and calls \texttt{output.writeUTF(toSort)} to write the unsorted string to servers input-stream. Then it waits until the server writes something to the clients input and reads it with \texttt{input.readUTF()}, to print it to the command-line. The client thread ends after that.\\
    \texttt{SimpleClientExecutor.java} is a class to start several clients and also to initiate the remote shut-down of the server. It also has a method \texttt{String getRandomString(int size)}, which creates a random string for a given length of the given alphabet (all letters + numbers). The main method of this class simply generates six instances of the \texttt{SimpleClient} class and starts them as a thread. Every instance is created with a different random string. The class also starts a specific thread, which connects to the server and sends the string \textit{-shutdown-}. This starts the shut-down process on the server side.\\
    \texttt{MultithreadedSortingServer.java} implements the \texttt{Runnable} interface and represents the server. The method \texttt{String sortString(String toSort)} uses Java 8 streams to sort a string. In the main method, this class starts an \texttt{ExecutorService} with a fixed thread-pool. Inside a \texttt{try} it starts a \texttt{ServerSocket} and accepts clients inside a \texttt{while(true)} loop. For each new client, a instance of \texttt{MultithreadedSortingServer} is submitted to thread-pool. This starts a new thread for each client. The \texttt{run()} method invokes the \texttt{reply()} method, which reads the input-stream (to get the clients unsorted string) and sorts the string. Then it writes it to the output.\\
    Although, if the \texttt{string.equals("-shutdown-")}, the request-handling thread calls \texttt{serverSocket.close()}. This triggers a \texttt{SocketException} at the main thread, which is calling \texttt{serverSocket.accept()}. The exception is caught, which will execute \texttt{shutdown()} on the executor-service. After that, all tasks are finished and the executor-service shuts down. This also ends the main server thread.\\
    
    \subsection*{run part1)}
    
      To execute the first part, two classes have to be run:
      \begin{itemize}
    	\item[1.] Run the main method of \texttt{MultithreadedSortingServer.java}
    	\item[2.] Run the main method of \texttt{SimpleClientExecutor.java}
      \end{itemize}
      Sample output:\\
      \textbf{MultithreadedSortingServer.java}:\\
      \begin{lstlisting}[language=sh]
        Accepted new Client!
        Accepted new Client!
        Accepted new Client!
        Accepted new Client!
        Accepted new Client!
        Accepted new Client!
        Accepted new Client!
        Shutting down server!
        Shutdown successful!
      
        Process finished with exit code 0
      \end{lstlisting}
      \textbf{SimpleClientExecutor.java}:\\
      \begin{lstlisting}[language=sh]
        Connected to server on port 8888.
        Connected to server on port 8888.
        Connected to server on port 8888.
        Connected to server on port 8888.
        Connected to server on port 8888.
        Connected to server on port 8888.
        Server changed 7szpefouap to 7aefoppsuz
        Server changed eruh19u3t3 to 1339ehrtuu
        Server changed 8b2mzip5rw to 258bimprwz
        Server changed d76yqgb2ri to 267bdgiqry
        Server changed mdvy4u6hc9 to 469cdhmuvy
        Server changed jk15mn5mpq to 155jkmmnpq
      
        Process finished with exit code 0
      \end{lstlisting}      
	
\end{document}