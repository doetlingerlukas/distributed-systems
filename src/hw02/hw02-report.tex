\documentclass{article}
\usepackage[T1]{fontenc}

\usepackage[utf8]{inputenc}
\usepackage[english]{babel}
\usepackage{upgreek}
\usepackage{amsmath}
\usepackage{color}
\usepackage{listings}
\usepackage{graphicx}

\begin{document}
  \title{Hw02 - Report - Distributed Systems}
  \date{}
  \author{Lukas Dötlinger, 01518316}
	
  \maketitle
  
  \section*{Introduction}
    
    All three parts of the homework sheet are completed.\\
    There is a subsection at the end of every part, which contains information on how to run the code. There is also a sample output of the program given.\\
	
  \section*{Part 1)}
	
    The first part is implemented using four Java classes, which are inside the \texttt{/part1} and \texttt{/utils} directories:
    \begin{itemize}
      \item \texttt{/part1/SimpleClient.java} represents a single client thread.
      \item \texttt{/part1/SimpleClientExecutor.java}, a class to start and execute client threads.
      \item \texttt{/part1/MultithreadedSortingServer.java} represents the server.
      \item \texttt{/utils/Protocol.java}, the utility class.
    \end{itemize}
    \texttt{SimpleClient.java} implements the \texttt{Runnable} interface and can therefore be started as thread. It has one private variable, \texttt{String toSort}, which is the string the client wants to be sorted by the server. The \texttt{void run()} method calls the \texttt{void request(String toSort)} method from the utility class \texttt{Protocol.java}.\\
    The utility request method gets the servers \texttt{Socket}. Address and port of the server are defined as \texttt{final} variables in \texttt{Protocol.java} and are accessible through getters. The method opens the input/output-streams and calls \texttt{output.writeUTF(toSort)} to write the unsorted string to servers input-stream. When the server writes something to the clients input, it reads it with \texttt{input.readUTF()}, to print it to the command-line.\\
    \texttt{SimpleClientExecutor.java} is a class to start several clients and also to initiate the remote shut-down of the server. It also has a method \texttt{String getRandomString(int size)}, which creates a random string for a given length of the given alphabet (all letters + numbers). The main method of this class simply generates six instances of the \texttt{SimpleClient} class and starts them as a thread. Every instance is created with a different random string. The class also starts a specific thread, which connects to the server and sends the string \textit{-shutdown-}. This starts the shut-down process on the server side.\\
    \texttt{MultithreadedSortingServer.java} implements the \texttt{Runnable} interface and represents the server. In the main method, this class starts an \texttt{ExecutorService} with a fixed thread-pool. Inside a \texttt{try} it starts a \texttt{ServerSocket} and accepts clients inside a \texttt{while(true)} loop. For each new client, a instance of \texttt{MultithreadedSortingServer} is submitted to thread-pool. This starts a new thread for each client. The \texttt{run()} method invokes the \texttt{void reply(Socket clientSocket, ServerSocket serverSocket)} method from the utility class \texttt{Protocol.java}, which reads the input-stream (to get the clients unsorted string) and sorts the string. Then it writes it to the output. The method \texttt{String sortString(String toSort)}, als located in the utility class, uses Java 8 streams to sort the string.\\
    Although, if the \texttt{string.equals("-shutdown-")}, the request-handling thread calls \texttt{serverSocket.close()}. This triggers a \texttt{SocketException} at the main server-thread, which is calling \texttt{serverSocket.accept()}. The exception is caught, which will execute \texttt{shutdown()} on the executor-service. After that, all tasks are finished and the executor-service shuts down. This also ends the main server thread.\\
    
    \subsection*{run part1)}
    
      To execute the first part, two classes have to be run:
      \begin{itemize}
    	\item[1.] Run the main method of \texttt{MultithreadedSortingServer.java}
    	\item[2.] Run the main method of \texttt{SimpleClientExecutor.java}
      \end{itemize}
      Sample output:\\
      \textbf{MultithreadedSortingServer.java}:\\
      \begin{lstlisting}[language=sh]
        Accepted new Client!
        Accepted new Client!
        Accepted new Client!
        Accepted new Client!
        Accepted new Client!
        Accepted new Client!
        Accepted new Client!
        Shutting down server!
        Shutdown successful!
      
        Process finished with exit code 0
      \end{lstlisting}
      \textbf{SimpleClientExecutor.java}:\\
      \begin{lstlisting}[language=sh]
        Connected to server on port 8888.
        Connected to server on port 8888.
        Connected to server on port 8888.
        Connected to server on port 8888.
        Connected to server on port 8888.
        Connected to server on port 8888.
        Server changed 7szpefouap to 7aefoppsuz
        Server changed eruh19u3t3 to 1339ehrtuu
        Server changed 8b2mzip5rw to 258bimprwz
        Server changed d76yqgb2ri to 267bdgiqry
        Server changed mdvy4u6hc9 to 469cdhmuvy
        Server changed jk15mn5mpq to 155jkmmnpq
      
        Process finished with exit code 0
      \end{lstlisting}     
      
  \section*{Part 2)} 
    
    The second part is implemented using six Java classes, which are inside the \texttt{/part1}, \texttt{/part2} and \texttt{/utils} directories:
    \begin{itemize}
      \item \texttt{/part1/SimpleClient.java} represents a single client thread.
      \item \texttt{/part2/ClientExecutor.java}, a class to start and execute client threads.
      \item \texttt{/part2/ProxyServer.java} represents the proxy-server.
      \item \texttt{/part2/EndPointSortingServer.java} represents an end-point-server.
      \item \texttt{/utils/Protocol.java}, the utility class.
      \item \texttt{/utils/ServerData.java} a class representing data of a end-point-server.
    \end{itemize}
    The \texttt{ClientExecutor.java} does the same as in part1. It starts some clients with a random string and also the shutdown thread.\\
    The multi-threaded \texttt{ProxyServer.java} implements the \texttt{Runnable} interface. In the main method, the proxy starts the end-point servers and adds an object of \texttt{ServerData} for each server to it's local server-list. \texttt{ServerData} stores the id, port and name of a server. After that, the proxy goes into a \texttt{while(true)} loop, where it is accepting clients. When a client connects, it submits a new instance of \texttt{ProxyServer} to it's executor-service. This thread will execute the \texttt{run()} method of the \texttt{ProxyServer.java} class, which reads the string \textit{toSort} from a client and writes it to a randomly chosen end-point-server. It then writes the answer from the end-point server to the client, attached also information on which server completed the task.\\
    \textbf{Fault tolerance}: If the randomly chosen server is not reachable, opening a socket to it will cause an \texttt{ConnectException}, which is caught. In the catch statement, the previously chosen server is removed from the list of servers. If the server-list is not empty, a recursive call, \texttt{run()}, is called. This will repeat the process of choosing a server.\\
    The \texttt{EndPointSortingServer.java} represents a multi-threaded end-point-server. I chose the server to have multiple threads with an executor-service. Therefore it is never unreachable because of being busy. It works similar to the server from part1. It implements the \texttt{Runnable} interface. Inside the \texttt{run()} method it accepts clients in a \texttt{while(true)} loop. Each client is handled by a new thread, submitted to an executor service. This thread uses the \texttt{void reply(Socket clientSocket, ServerSocket serverSocket)} from the utility class \texttt{Protocol.java} to answer to a client. Since the server is multi-threaded it can easily be closed remotely like in part1. 
    
    \subsection*{run part2)}
      
      To execute the second part, two classes have to be run:
      \begin{itemize}
      	\item[1.] Run the main method of \texttt{ProxyServer.java}
      	\item[2.] Run the main method of \texttt{ClientExecutor.java}
      \end{itemize}
      Sample output:\\
      \textbf{ProxyServer.java}:\\
      \begin{lstlisting}[language=sh]
        Started EndPointServer 3!
        Started EndPointServer 2!
        Started EndPointServer 4!
        Started EndPointServer 1!
        Proxy accepted new Client!
        Proxy accepted new Client!
        Proxy accepted new Client!
        Proxy accepted new Client!
        Proxy accepted new Client!
        Proxy accepted new Client!
        Proxy accepted new Client!
        EndPointServer 1 accepted new task!
        EndPointServer 4 accepted new task!
        EndPointServer 4 accepted new task!
        EndPointServer 4 accepted new task!
        EndPointServer 4 shutting down!
        EndPointServer 4 closed!
        EndPointServer 2 accepted new task!
        EndPointServer 2 accepted new task!
        Server 4 not available! Redirecting..
        EndPointServer 2 accepted new task!
      \end{lstlisting}
      \textbf{ClientExecutor.java}:\\
      \begin{lstlisting}[language=sh]
        Connected to server on port 8888.
        Connected to server on port 8888.
        Connected to server on port 8888.
        Connected to server on port 8888.
        Connected to server on port 8888.
        Connected to server on port 8888.
        Server changed tlf9auas6z to 69aaflstuz with end-point server 4
        Server changed ksidzu7tvx to 7dikstuvxz with end-point server 1
        Server changed jth741faj9 to 1479afhjjt with end-point server 4
        Server changed 6e8eo1xrmv to 168eemorvx with end-point server 2
        Server changed ubhtqyolgv to bghloqtuvy with end-point server 2
        Server changed as5e3gkiwj to 35aegijksw with end-point server 2
        
        Process finished with exit code 0
      \end{lstlisting}
      
  \section*{Part 3)}
  
    The third part is an extension to part2. Reimplements two classes:
    \begin{itemize}
      \item \texttt{/part3/AdvancedProxyServer.java} is a reimplementation of the previous proxy.
      \item \texttt{/part3/RegisteringEndPointServer.java} is a reimplementation of the previous end-point-server.
    \end{itemize}
	The difference to the previous proxy-server is that \texttt{AdvancedProxyServer.java} doesn't add the end-point-servers to the list while starting them. Although the proxy still starts the servers, it is not necessary, since they are registering automatically (this is only done so that one doesn't have to execute more than two java classes).\\
	A \texttt{RegisteringEndPointServer} registers itself (opens a socket connection) automatically to the proxy be sending an predefined string, \textit{-registration-}, on which the proxy answers with \textit{-OK-}. After that, the server will send it's port to the proxy. Every end-point-server trys to register to the proxy, before entering the \texttt{while(true)} loop to accept clients. This is the only difference to the end-point-server of part2.\\
	Before giving a client socket to a request handling thread, the proxy reads the input string to check whether it is a registering server or a requesting client. Registrations are directly handled by the main proxy thread, to update it's list of available services, which is passed to the client-request handling threads.\\
	The whole client implementation stays the same.
  
    \subsection*{run part3)}
      
      To execute the third part, two classes have to be run:
      \begin{itemize}
      	\item[1.] Run the main method of \texttt{/part3/AdvancedProxyServer.java}
      	\item[2.] Run the main method of \texttt{/part2/ClientExecutor.java}
      \end{itemize}
      Sample output:\\
      \textbf{AdvancedProxyServer.java}:\\
      \begin{lstlisting}[language=sh]
        Proxy accepted new Server on port 8001!
        Proxy accepted new Server on port 8004!
        Started EndPointServer 4!
        Proxy accepted new Server on port 8002!
        Started EndPointServer 2!
        Proxy accepted new Server on port 8003!
        Started EndPointServer 3!
        Started EndPointServer 1!
        Proxy accepted new Client!
        Proxy accepted new Client!
        Proxy accepted new Client!
        Proxy accepted new Client!
        Proxy accepted new Client!
        Proxy accepted new Client!
        Proxy accepted new Client!
        EndPointServer 4 accepted new task!
        EndPointServer 1 accepted new task!
        EndPointServer 3 accepted new task!
        EndPointServer 4 accepted new task!
        EndPointServer 4 shutting down!
        EndPointServer 4 closed!
        EndPointServer 2 accepted new task!
        Server 4 not available! Redirecting..
        Server 4 not available! Redirecting..
        EndPointServer 1 accepted new task!
        EndPointServer 3 accepted new task!
      \end{lstlisting}
      \textbf{ClientExecutor.java}:\\
      \begin{lstlisting}[language=sh]
        Connected to server on port 8888.
        Connected to server on port 8888.
        Connected to server on port 8888.
        Connected to server on port 8888.
        Connected to server on port 8888.
        Connected to server on port 8888.
        Server changed klti3ba79i to 379abiiklt with end-point server 1
        Server changed 9ay2xw1g15 to 11259agwxy with end-point server 4
        Server changed 9q1e49xt9m to 14999emqtx with end-point server 3
        Server changed zwpojest7a to 7aejopstwz with end-point server 2
        Server changed rfhrx6nxaf to 6affhnrrxx with end-point server 3
        Server changed eltz1lu4k6 to 146eklltuz with end-point server 1
        
        Process finished with exit code 0
      \end{lstlisting}
	
\end{document}